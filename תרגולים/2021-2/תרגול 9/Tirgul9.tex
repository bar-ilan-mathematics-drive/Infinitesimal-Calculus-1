\documentclass{article}
\usepackage{fontspec, fullpage}
\usepackage{polyglossia}
\usepackage{amsmath, amssymb, bbm, amsthm, graphicx}
\setmainlanguage{hebrew}
\setmainfont{Times New Roman}
% \newfontfamily{\hebrewfont}{New Peninim MT}
\begin{document}
\title{תרגול 9 חשבון אינפיניטסימלי 1 שנת 2021/2}
\author{ישראל הבר}
\maketitle

\newtheorem{theorem}{משפט}
\newtheorem{exercise}{תרגיל}
\newtheorem{homeexercise}{תרגיל לבית}
\newtheorem{example}{דוגמה}
\theoremstyle{definition}
\newtheorem{definition}{הגדרה}
\newtheorem{notation}{סימון}
\newtheorem{claim}{טענה}
\newtheorem{comment}{\emph{הערה}}
\renewcommand\qedsymbol{$\blacksquare$}
\newcommand{\limtoinfty}{\underset{n\rightarrow\infty}{\lim}}
\newcommand{\limtur}{\overset{\infty}{\underset{n=1}{\sum}}}
\newcommand{\limturstart}[1]{\overset{\infty}{\underset{n=#1}{\sum}}}
\newcommand{\limsuptoinfty}{\underset{n\rightarrow\infty}{\limsup}}
\newcommand{\liminftoinfty}{\underset{n\rightarrow\infty}{\liminf}}
\newcommand{\limtoinftym}{\underset{m\rightarrow\infty}{\lim}}
\newcommand{\limtop}{\underset{-}{\lim}}
\newcommand{\limbottom}{\overset{-}{\lim}}
\newcommand{\goesto}{\underset{n\rightarrow\infty}{\longrightarrow}}
\newcommand{\goestom}{\underset{m\rightarrow\infty}{\longrightarrow}}
\newcommand{\goesfrom}{\underset{n\rightarrow\infty}{\longleftarrow}}
\newcommand{\funclim}[2]{\underset{#1\rightarrow#2}{\lim}\,}
% \newcommand{\series}{_{n\in\mathbb{N}}}
\newcommand{\series}[2]{\{#1\}_{#2\in\mathbb{N}}}

\begin{theorem}
תהי פונקציה 
$f$
המוגדרת בסביסה מנוקבת 
$A$
של 
$a$.
\par\noindent
$L\in\mathbb{R}$
נקרא הגבול של 
$f$
בנקודה 
$a$
אם:
\[\forall\epsilon>0\quad\exists\delta>0\quad \forall x\in A: \quad 0<|x-a|<\delta\longrightarrow |f(x)-L|<\epsilon\]
ונסמן 
\[\funclim{x}{a} f(x) = L\]
\end{theorem}

\begin{comment}
גבול לפי קושי שקול לגבול לפי היינה.
\end{comment}

\begin{definition}
עוד הגדרות של גבולות - 
\begin{enumerate}
\item נאמר ש
$\funclim{x}{a} = -\infty$
בסביבה מנוקבת 
$A$
של 
$a$
אם 
\[\forall m<0\quad\exists \delta>0\quad\forall x\in A:\quad 0<|x-a|<\delta\longrightarrow f(x)<m\]
\item נאמר כי 
$\funclim{x}{\infty} f(x) = L$
אם 
\[\forall\epsilon>0\quad\exists M>0:\quad x>M \longrightarrow |f(x)-L|<\epsilon\]
\item 
נאמר כי 
$\funclim{x}{-\infty} f(x) = L$
אם 
\[\forall\epsilon>0\quad\exists M>0:\quad x<M \longrightarrow |f(x)-L|<\epsilon\]
\end{enumerate}
\end{definition}

\begin{exercise}
הוכיחו לפי הגדרה 
\[\funclim{x}{-\infty}\frac{x+5}{2x+3}=\frac{1}{2}\]
\end{exercise}

\begin{proof}
\[\forall\epsilon>0\quad \exists m<0: \quad x<m\longrightarrow\left|f(x)-\frac{1}{2}\right|<\epsilon\]

יהי 
$\epsilon>0$
נשים לב כי אנחנו רוצים 
\[\left|\frac{x+5}{2x+3}-\frac{1}{2}\right|= \left|\frac{7}{4x+6}\right|<\epsilon\]
אנחנו רוצים שאם 
$x<m$
יתקיים הביטוי הקודם. אם מקטינים את האיקס במשוואה נקבל שהביטוי קטן ולכן עבור 
$m<-\frac{6}{4}$
יתקיים שצריך שיתקיים 
\[\frac{7}{-4m-6}<\epsilon\]
וזה יהיה שקול ל 
\[m<-\frac{1}{4}\left(\frac{7}{\epsilon}+6\right)\]
ולכן סך הכל ניקח 
\[m<\min\left\{-\frac{1}{4}\left(\frac{7}{\epsilon} +6\right), -\frac{3}{2}\right\}\]
\end{proof}

\begin{exercise}
הוכיחו לפי הגדרה 
\[\funclim{x}{1}\frac{x^2+1}{(x-1)^2}=\infty\]
\end{exercise}

\begin{proof}
צריך להראות כי 
\[\forall M>0\quad \exists \delta>0:\quad 0<|x-1|<\delta\longrightarrow f(x)>M\]
לכן יהי 
$M>0$
ונחפש 
$\delta>0$
מתאים. 
\[\frac{x^2+1}{(x-1)^2}>M\]
אך נשים לב כי אם 
$0<|x-1|<\delta$ 
יתקיים כי 
$|x-1|^2<\delta^2$
ולכן 
\[f(x)>\frac{x^2+1}{\delta^2}>\frac{1}{\delta^2}\]
ונרצה למצוא 
$\delta>0$
כך ש 
$f(x)> M$
ולכן ניקח 
\[\delta<\frac{1}{\sqrt{M}}\]
\end{proof}

\begin{exercise}
חשבו 
\[\funclim{x}{2} \frac{x^3-8}{x-2}\]
\end{exercise}

\begin{proof}
נשים לב כי 
\[x^3-8 = (x-2)(x^2+2x+4)\]
ולכן עבור 
$x\neq 2$
יתקיים 
\[\frac{x^3-8}{x-2} = x^2+2x+4\]
ולכן מאריתמטיקת של גבולות נקבל כי הגבול הוא 12.
\end{proof}

\section{גבולות חד צדדיים}
נשים לב כי למשל עבור 
$f(x)=\sqrt{x}$
מוגדרת רק עבור 
$x\geq 0$
ולכן אם למשל נרצה להגדיר גבול של הפונקציה ב0 נצטרך הגדרות חדשות - לפי קושי 
\begin{enumerate}
\item 
$\funclim{x}{x_0^{+}}f(x) = L$
אם 
\[\forall\epsilon>0\quad\exists\delta>0:\quad 0<x-x_0<\delta\longrightarrow |f(x)-L|<\epsilon\]
\item $\funclim{x}{x_0^{+}}f(x) = L$
אם 
\[\forall\epsilon>0\quad\exists\delta>0:\quad 0<x_0-x<\delta\longrightarrow |f(x)-L|<\epsilon\]
\end{enumerate}

\begin{exercise}
חקרו את הגבולות החד צדדיים של 
\[f(x)=\frac{\sqrt{x^2-9}}{|x-3|}\text{ עבור }x_0=3\]
\end{exercise}

\begin{proof}
נשים לב כי עבור סביבה מנוקבת שמאלית של 3 הפונקציה לא מוגדרת ולכן אין עבורה גבול חד צדדי. נחשב עבור סביבה מנוקבת ימנית 
\[\funclim{x}{3^+}\frac{\sqrt{x^2-9}}{|x-3|} = \funclim{x}{3^+}\frac{\sqrt{x+3}}{\sqrt{x-3}} = \infty \left("=\frac{6}{\infty}"\right)\]
\end{proof}

\newpage

\begin{theorem}
הגבול של הפונקציה בנקודה קיים אם ורק אם הגבולות החד צדדיים קיימים ושווים.
\end{theorem}

\begin{exercise}
חשבו את 
$\funclim{x}{0} \sin(x)$
\end{exercise}

\begin{proof}
קודם נתבונן בציור הבא - 
\begin{center}
\includegraphics[scale=0.5]{./images/SimpleSinGraph}
\end{center}
כאשר הגובה הוא פונקציית סינוס והקשת היא 
$x$. 
ניתן לראות מכאן כי עבור התחום 
$0<x<\frac{1}{2}\pi$
נקבל כי 
\[0<\sin(x)<x\]
ולכן מסנדוויץ' נקבל כי 
\[\funclim{x}{0^+} \sin(x) = 0\]
עכשיו עבור התחום 
$-\frac{\pi}{2}<x<0$
נקבל כי 
$\sin(-x)=-\sin(x)$
ולכן בחישוב הגבול נקבל גם כי 
\[\funclim{x}{0^-} \sin(x) = 0\]
סך הכל נקבל כי 
\[\funclim{x}{0}\sin(x) = 0\]
\end{proof}

\begin{exercise}
חשבו 
$\funclim{x}{0}\frac{\sin(x)}{x}$
\end{exercise}

\begin{proof}
נצייר מחדש עם הוספה חשובה 
\begin{center}
\includegraphics[scale=0.6]{./images/NextSinGraph}
\end{center}
נשים לב כי זה גורר שבתחום 
$0<x<\frac{1}{2}\pi$
יתקיים אז 
\[\frac{1}{2}\sin(x)<\frac{1}{2}\sin(x)<\frac{1}{2}\tan(x)\]
עבור הגבול החד צדדי הימני נקבל כי ניתן לחלק ב 
$\sin(x)$
ולכן 
\[1<\frac{x}{\sin x}<\frac{1}{\cos x}\]
ולכן לפי סנדוויץ' נקבל כי 
\[\funclim{x}{0^+}\frac{x}{\sin x} = 1\]
נשים לב כי הפונקציה זוגית ולכן נקבל כי 
\[\funclim{x}{0^-}\frac{x}{\sin x} = 1\]
בפרט הגבול קיים ושווה 1.
ולכן גם עבור המקורי הגבול יהיה 1.
\end{proof}

\section{רציפות}
\begin{definition}
פונקציה 
$f$
רציפה בנקודה 
$x_0\in\mathbb{R}$
אם הגבול בנקודה קיים ומקיים 
\[f(x_0) = \funclim{x}{x_0} f(x)\]
\end{definition}

\begin{exercise}
הוכיחו כי פונקציית הסינוס רציפה בכל נקודה.
\end{exercise}

\begin{proof}
נשים לב כי זה שקול להראות ש 
\[\funclim{h}{0} \sin(x_0+h) = \sin(x_0)\]
נפתח את הביטוי מימין - 
\[\sin(x_0+h) = \sin(x_0)\cos(h) + \sin(h)\cos(x_0)\]
כשנפעיל את הקבול המתאים נקבל כי 
\[\funclim{h}{0}\sin(x_0+h) = \sin x_0\cdot 1+\cos x_0\cdot 0 = \sin(x_0) \]
לכן הפונקציה רציפה לכל נקודה 
$x_0\in\mathbb{R}$.
\end{proof}

\begin{exercise}
מצאו 
$a,b\in\mathbb{R}$
כך ש 
\[f(x) = \begin{cases}-2\sin x,& x\leq-\frac{1}{2}\pi \\ a\sin x + b,& -\frac{\pi}{2}<x\leq\frac{1}{2}\pi \\ \cos x,& x>\frac{1}{2}\pi\end{cases}\]
רציפה בכל נקודה.
\end{exercise}

\begin{proof}
נשים לב כי בתוך התחומים האלה ממש הפונקציות רציפות. לכן ק' צריך לוודא עבור "נקודות החיתוך". נשים לב כי 

\begin{align*}
\funclim{x}{-\left(\frac{1}{2}\pi\right)^-} f(x) &= -2\sin\left(-\frac{1}{2}\pi\right)=2 \\
\funclim{x}{\left(-\frac{1}{2}\pi\right)^+} f(x) &= a\sin\left(-\frac{\pi}{2}\right) + b = b-a \\
\funclim{x}{\left(\frac{1}{2}\pi\right)^-} f(x) &= a\sin\left(\frac{\pi}{2}\right) + b = a+b \\
\funclim{x}{\left(\frac{1}{2}\pi\right)^+} f(x) = 0
\end{align*}
מזה נקבל שהפונקציה רציפה אם ורק אם מערכת המשוואות הבאה מתקיימת 
\[\begin{cases} b-a&=2 \\ a+b&=0\end{cases}\]
כלומר אם ורק אם 
\[a=-1, \quad b=1 \]
\end{proof}

\end{document}
