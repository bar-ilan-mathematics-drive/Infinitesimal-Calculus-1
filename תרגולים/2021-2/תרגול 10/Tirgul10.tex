\documentclass{article}
\usepackage{fontspec, fullpage}
\usepackage{polyglossia}
\usepackage{amsmath, amssymb, bbm, amsthm, graphicx}
\setmainlanguage{hebrew}
\setmainfont{Times New Roman}
% \newfontfamily{\hebrewfont}{New Peninim MT}
\begin{document}
\title{תרגול 10 חשבון אינפיניטסימלי 1 שנת 2021/2}
\author{ישראל הבר}
\maketitle

\newtheorem{theorem}{משפט}
\newtheorem{lemma}{למה}
\newtheorem{exercise}{תרגיל}
\newtheorem{homeexercise}{תרגיל לבית}
\newtheorem{example}{דוגמה}
\theoremstyle{definition}
\newtheorem{definition}{הגדרה}
\newtheorem{notation}{סימון}
\newtheorem{claim}{טענה}
\newtheorem{comment}{\emph{הערה}}
\renewcommand\qedsymbol{$\blacksquare$}
\newcommand{\limtoinfty}{\underset{n\rightarrow\infty}{\lim}}
\newcommand{\limtur}{\overset{\infty}{\underset{n=1}{\sum}}}
\newcommand{\limturstart}[1]{\overset{\infty}{\underset{n=#1}{\sum}}}
\newcommand{\limsuptoinfty}{\underset{n\rightarrow\infty}{\limsup}}
\newcommand{\liminftoinfty}{\underset{n\rightarrow\infty}{\liminf}}
\newcommand{\limtoinftym}{\underset{m\rightarrow\infty}{\lim}}
\newcommand{\limtop}{\underset{-}{\lim}}
\newcommand{\limbottom}{\overset{-}{\lim}}
\newcommand{\goesto}{\underset{n\rightarrow\infty}{\longrightarrow}}
\newcommand{\goestom}{\underset{m\rightarrow\infty}{\longrightarrow}}
\newcommand{\goesfrom}{\underset{n\rightarrow\infty}{\longleftarrow}}
\newcommand{\funclim}[2]{\underset{#1\rightarrow#2}{\lim}\,}
% \newcommand{\series}{_{n\in\mathbb{N}}}
\newcommand{\series}[2]{\{#1\}_{#2\in\mathbb{N}}}

\begin{exercise}
\textbf{הוכח או הפרך:}
\begin{enumerate}
\item אם 
$f\neq g$
לא רציפות בנקודה 
$x_0$
אז גם 
$f(x)+g(x)$
לא רציפה בנקודה זו.
\item אם 
$f\neq g$
לא רציפות בנקודה 
$x_0$
אז גם 
$f(x)\cdot g(x)$
\end{enumerate}
\end{exercise}

\begin{proof}
\begin{enumerate}
\item \textbf{הפרכה} \par\noindent
ניקח למשל 
\[f(x)=\begin{cases}\frac{1}{x}, & x\neq 0 \\ 1, & x=0\end{cases},\quad g(x) = -f(x)\]
נשים לב כי 2 הפונקציות הנל לא רציפות בבנקודה 
$x_0=0$
אך 
\[g(x)+f(x) = 0\]
רציפה בנקודה זו.
\item \textbf{הפרכה}\par \noindent
ניקח 
\[f(x):=\begin{cases}1, &x\neq 3\\ -1, &x=3\end{cases},\quad g(x):=\begin{cases}-1, & x\neq 3 \\ 1, & x=3 \end{cases}\]
\end{enumerate}
\end{proof}

\section{מיון נקודות אי רציפות}
יש כמה אפשרויות ללמה הפונקציה לא רציפה בנקודה מסויימת - 
\begin{itemize}
\item 
$a$
\textbf{נקודת אי רציפת סליקה: }
\par\noindent
הגבול קיים אך הערך של הפונקציה לא שווה לגבול בנקודה זו (או שהפונקציה לא מוגרת בנקודה זו.)
\item 
$a$
\textbf{נקודת אי רציפות קפיצה: }
\par\noindent
אם הגבולות החד צדדיים קיימים אך שונים. 
\item 
$a$
\textbf{נקודת אי רציפות מסוג שני: }
\par \noindent
לפחות אחד מהגבולות החד צדדיים לא קיים.
\end{itemize}

\begin{exercise}
מיינו את נקודות האי רציפות של הפונקציה הבאה 
\[f(x) = \frac{\frac{1}{x} - \frac{1}{x+1}}{\frac{1}{x-1}-\frac{1}{x}}\]
\end{exercise}

\begin{proof}
נשים לב כי (כאשר הכל מוגדר)
\[f(x) = \frac{\frac{1}{x} - \frac{1}{x+1}}{\frac{1}{x-1}-\frac{1}{x}} = \frac{(x-1)x}{x(x+1)}\]
עכשיו צריך לאפיין את הנקודות הרגישות. נשים לב כי בהגדרת הפונקציה הנקודות האלה הן 
$x\in\{0, \pm 1\}$. 
\begin{itemize}
\item $\textbf{x=0:}$ \par\noindent
נשים לב כי 
\[\funclim{x}{0}\frac{x(x-1)}{x(x+1)} = -1\]
ולכן הגבולות החד צדדיים קיימים ושווים, מכיוון שהפונקציה אינה מוגדרת בנקודה זו נקבל שזאת נקודה אי רציפות סליקה.
\item $\textbf{x=1:}$ \par\noindent
נשים לב כי 
\[\funclim{x}{1}\frac{x(x-1)}{x(x+1)} = 0\]
מאותן סיבות זו נקודת אי רציפות סליקה. 
\item $\textbf{x=-1:}$ \par\noindent
נשים לב כי 
\begin{align*}
\funclim{x}{-1^-} f(x) &= -\frac{2}{0} = \infty \\
\funclim{x}{-1^+} f(x) &= \frac{-2}{0} = -\infty
\end{align*}
ובפרט הגבולות החד צדדיים לא קיימים ולכן זו נקודת אי רציפות מסוג שני. 
\end{itemize}
\end{proof}

\begin{exercise}
מיינו את נקודת האי-רציפות של הפונקציה 
\[f(x):=\sin\left(\frac{1}{\log(x^2)}\right)\]
\end{exercise}

\begin{proof}
צריך לאפיין את נקודות הרגישות של הפונקציה. נשים לב כי יכל להיות שנציב 0 בלוג שזה לא מוגדר, או שנקבל 
$\log(x^2) = 0$. 
לכן סך הכל הנקודות הבעייתיות הן 
$x = \pm 1$
לכן נחלק למקרים המתאימים 
\begin{itemize}
\item $\textbf{x=0:}$
\[\funclim{x}{0} \log x^2 = -\infty \Rightarrow \funclim{x}{0} \frac{1}{\log x^2} = 0 \Rightarrow \funclim{x}{0} \sin\left(\frac{1}{\log(x^2)}\right) = 0\]
ולכן הגבולות החד צדדיים קיימים ושווים אך הפונקציה אינה מוגדרת בנקודה ולכן זאת נקודת אי רציפות סליקה. 
\item $\textbf{x=1:}$
ניקח את הסדרות הבאות - 
\begin{align*}
x_n &= \sqrt{e^{1/\left(0.5\pi +2\pi n\right)}} \\
y_n &= \sqrt{e^{1/\left(-0.5\pi +2\pi n\right)}}
\end{align*}
נשים לב כי הסדרות האלה מתכנסות ל1 אך 
\[f(x_n)\goesto 1,\quad f(y_n)\goesto -1\]
ולכן הגבול החד צדדי לא קיים ובפרט זו נקודת אי רציפות מסוג 2. בדומה עבור המקרה 
$x=-1$.
\end{itemize}
\end{proof}
\newpage
\begin{theorem}[ערך הביניים]
תהי 
$f$
פונקציה רציפה בקטע 
$[a, b]$
ויהי 
$y_0$
ערך המקיים 
$f(x)\leq y_0\leq f(b)$.
קיימת 
$x_0\in[a, b]$
כך ש 
$f(x_0)=y_0$
\end{theorem}

\begin{exercise}
תהי 
$f$
פונקציה רציפה ב 
$[0,2]$
כך ש 
$f(2)=3$.
הוכח שקיים 
$x_0\in [0, 2]$
כך ש 
$f(x_0)=\frac{1}{x_0}$
\end{exercise}

\begin{proof}
נגדיר 
$h(x)=x f(x)$.
נשים לב כי 
$h(0)=0, h(2)=6$. 
הפונקציה רציפה בקטע ולכן לפי משפט ערך הביניים יש ערך 
$x_0$
כך ש 
$h(x_0)=1$. 
ולכן 
$f(x_0)x_0 = 1$
ולכן 
\[f(x_0)=\frac{1}{x_0}\]
\end{proof}

\begin{exercise}
תהיינה 
$f,g$
פונקציות רציפות בקטע 
$[0, 1]$
המקיימות 
\[g([0, 1])=[0, 1], \quad f([0, 1])\subseteq [0,1]\]
הוכיחו כי קיימת נקודה 
$x_0$
כך ש 
$f(x_0)=g(x_0)$.
\end{exercise}

\begin{proof}
נגדיר 
$h(x):=f(x)-g(x)$. 
$h$
רציפה ב 
$[0, 1]$
כחיסור של רציפות. קיימים 
$x_1\neq x_2$
כך ש 
$g(x_1)=1, g(x_2)=0$. 
נשים לב כי מהנתון 
$f(x_2)\leq 1, f(x_1)\geq 0$. 
ולכן 
\begin{align*}
h(x_1)&=f(x_1)-g(x_1)=f(x_1)-1\leq 0\\
h(x_2)&=f(x_2)-g(x_2)=f(x_2)-0\geq 0 
\end{align*}
בלי הגבלת הכלליות 
$x_1<x_2$.
$h$
רציפה ב 
$[x_1, x_2]$. 
ולכן לפי משפט ערך הביניים יש 
$x_0\in [x_1,x_2]\subseteq [0, 1]$
כך ש - 
\[h(x_0)=0\Rightarrow f(x_0)-g(x_0) = 0\Rightarrow f(x_0)=g(x_0)\]
\end{proof}

\begin{theorem}[ויירשטראס]
פונקציה רציפה בקטע סגור מקבלת בו מינימום ומקסימום.
\end{theorem}

\begin{exercise}
נתונות 2 פונקציות 
$f,g$
רציפות ב 
$[0, 1]$
ומתקיים 
\[\sup\{f(x)\mid x\in [0, 1]\} = \sup\{g(x)\mid x\in [0, 1]\}\]
הוכיחו שקיימת נקודה 
$x\in [0, 1]$
 כך ש- 
$g(x_0)=f(x_0)$.
\end{exercise}

\begin{proof}
לפי ויירשטראס הסופרימום הנ"ל קיים ולכן נסמן אותו ב
$M$. 
\[M=\underset{x\in[0, 1]}{\sup \{f(x)\}}=\underset{x\in[0, 1]}{\sup \{g(x)\}}\]
כאשר קיימות 
$x_1, x_2\in [0, 1]$
כך ש- 
\[f(x_1)=M, g(x_2)=M\]
נגדיר 
$h(x):=f(x)-g(x)$.
נשים לב כי 
$h$
רציפה ב- 
$[0, 1]$
כחיסור רציפות. לכן 
\begin{align*}
h(x_1)&= f(x_1)-g(x_1) = M-g(x_1)\geq 0 \\
h(x_2)&= f(x_2)-g(x_2) = f(x_2)-M\leq 0 
\end{align*}
בדומה לתרגיל הקודם יהיה 
$x_1<x_0<x_2$
כך ש 
$h(x_0)=0$.
ולכן נקבל 
$f(x_0)=g(x_0)$.
\end{proof}

\begin{lemma}
תהי 
$f:X\longrightarrow Y$
פונקציה עולה ממש, רציפה ועל. לכל נקודה 
$x\in X$
שהיא נקודת הצטברות מימין/משמאל של הקבוצה 
$X$
אזי הנקודה 
$y:=f(x)\in Y$
היא נקודת הצטברות מימין/משמאל של הקבוצה 
$Y$
בהתאמה. 
\end{lemma}

\begin{proof}
נוכיח עבור גבול משמאל ונשאיר את המקרה עבור הגבול מימין שהוא על אותו עיקרון. נניח שיש סדרה עולה ממש ב 
$X$
- 
$\series{x_n}{n}$
כך ש 
$x_n\goesto x$. 
מרציפות הפונקציה 
$f(x_n)\goesto f(x) = y$. 
כיוון ש 
$f$
עולה ממש, אזי 
$f(x_n)<y$
לכל 
$n\in\mathbb{N}$.
ולכן 
$y$
נקודת הצטברות משמאל של 
$Y$.
\end{proof}












\end{document}

