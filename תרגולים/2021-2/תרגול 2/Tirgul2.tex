\documentclass{article}
\usepackage{fontspec, fullpage}
\usepackage{polyglossia}
\usepackage{amsmath, amssymb, bbm, amsthm}
\setmainlanguage{hebrew}
\setmainfont{Times New Roman}
% \newfontfamily{\hebrewfont}{New Peninim MT}
\begin{document}
\title{תרגול 2 חשבון אינפיניטסימלי 1 שנת 2021/2}
\author{ישראל הבר}
\maketitle

\newtheorem{theorem}{משפט}
\newtheorem{exercise}{תרגיל}
\newtheorem{homeexercise}{תרגיל לבית}
\newtheorem{solution}{Solution Of}
\theoremstyle{definition}
\newtheorem{definition}{הגדרה}
\newtheorem{notation}{סימון}
\newtheorem{claim}{טענה}
\renewcommand\qedsymbol{$\blacksquare$}

\begin{definition}
סדרה זאת פונקציה מהטבעיים לקבוצה אחרת. בקורס הזה זה יהיה כמעט אקסקלוסיבית לממשיים. כלומר סדרה זה פונקציה
\[f:\mathbb{N}\rightarrow\mathbb{R}\]
כאשר בדרך כלל כותבים
\[f(n)=f_n\text{ or } a_n\]
\end{definition}

\begin{notation}
בדרך כלל אנחנו כותבים סדרה באופן הבא 
\[\left( a_n\right)_{n=1}^{\infty} \text{ or } \{a_n\}_{n=1}^{\infty}\]
\end{notation}

אינטואיטיבית אם אנחנו רואים סדרה אינסופית של מספרים אנחנו יכולים לראות אם המספרים מתקרבים למספר אחר או לא. למשל אם נראה את הסדרה הבא - 
\[1, \frac{1}{2}, \frac{1}{3}, \frac{1}{4}, \frac{1}{5},\dots,\frac{1}{10^{100}},\dots\]
מתקרב לאפס בצורה אינסופית, כלומר ככל שנמשיך זה ימשיך להתקרב לאפס. גם כמו שראיתם למשל בבגרות 
\[1, 1+\frac{1}{2}, 1+\frac{1}{2} + \frac{1}{4}, 1+\frac{1}{2} + \frac{1}{4} + \frac{1}{8}, 1+\frac{1}{2} + \frac{1}{4} + \frac{1}{8} + \frac{1}{16},  1+\frac{1}{2} + \frac{1}{4} + \frac{1}{8} + \frac{1}{16} + \frac{1}{32}, \dots\]
מתכנס ל2. אם נשנה את ה5 איברים הראשונים למשל המספרים עדיין יתקרבו ל2 כמה שנרצה. זה נכון גם אם אם נשנה את ה
$10^{100}$
מספרים הראשונים. בנוסף לכך ברור שהסדרה הקודמת לא שואפת ל 2.001 וזה בגלל שאנחנו לא נתקרב כמה שנרצה ל2.001. ההבנה הזאת יוצרת את ההגדרה הבאה עבור הגבול - 

\begin{definition}
נאמר שהסדרה 
$\{a_n\}_{n=1}^{\infty}$
שואפת/מתכנסת ל 
$L\in\mathbb{R}$
אם התנאי הבא מתקיים 
\[\forall\epsilon>0\,\exists N\in\mathbb{N}\,\forall n\geq N: |a_n-L|\leq \epsilon\]
\end{definition}

\begin{notation}
אם 
$(a_n)$
שואפת לL ניתן לסמן את זה גם בצורות הבאות
\[\underset{n\rightarrow\infty}{\lim a_n} = L\text{ or ,}\; a_n\underset{n\rightarrow\infty}{\longrightarrow}L\]
\end{notation}

כמובן שאם יש סדרה שמתקרבת בצורה אינסופית למספר אחד היא לא יכולה גם להתקרב אינסופית למספר אחר. מזה מקבלים את המשפט הבא - 

\begin{theorem}
אם לסדרה יש גבול, הוא יחיד.
\end{theorem}

\begin{exercise}
מצאו את הגבול הבא אם הוא קיים (והוכיחו שהוא קיים) - 
\[\underset{n\rightarrow\infty}{\lim} \frac{n-1}{n}\]
\end{exercise}

\begin{proof}
ניתן לראות שזה אמור להתקרב ל1, לכן נוכיח שהגבול הוא 1. בשביל זה קודם צריך לקחת 
$\epsilon>0$ 
ונוכיח שקיים 
$N\in\mathbb{N}$
כך שלכל 
$n\geq N$
מתקיים 
\begin{align*}
|a_n-1|&\leq \epsilon \\
\left|\frac{n-1}{n} - 1\right|&\leq\epsilon \\
\left|-\frac{1}{n}\right| &\leq \epsilon\\
n&\geq\frac{1}{\epsilon} 
\end{align*}

לכן ניקח למשל 
\[ N = \left\lceil\frac{1}{\epsilon}\right\rceil\]

\end{proof}

\begin{exercise}
הוכיחו לפי הגדרה ש
\[\underset{n\rightarrow\infty}{\lim}\:\frac{n^2-n-1}{3n^2+2n+1} = \frac{1}{3}\]
\end{exercise}

\begin{proof}
יהי 
$\epsilon>0$
ונוכיח שקיים 
$N\in\mathbb{N}$
כך שלכל 
$n\geq N$ 
מתקיים 
\begin{align*}
\left|\frac{n^2-n-1}{3n^2+2n+1} - \frac{1}{3}\right|&\leq\epsilon\\
\left| \frac{n^2-n-1-\frac{1}{3}(3n^2+2n+1)}{3n^2+2n+1} \right|&\leq \epsilon\\
\left|\frac{-n-1-\frac{2}{3}n-\frac{1}{3}}{3n^2+2n+1}\right|\leq\epsilon \\
\left|\frac{-\frac{5}{3}n -\frac{4}{3}}{3n^2+2n+1}\right|&\leq\epsilon
\end{align*}

נשים לב כי

\[\left|-\frac{5n+4}{3(3n^2+2n+1)}\right| \leq \frac{5n+4}{9n^2} \leq \frac{9n+9}{9n^2} = \frac{n+1}{n^2}\leq \frac{2n}{n^2} = \frac{2}{n}\]

לכן אם ניקח N כך ש 
$\frac{2}{N} \leq \epsilon$
נסיים. זה קורה כאשר לוקחים 
\[N=\left\lceil\frac{2}{\epsilon}\right\rceil\]
\end{proof}

\begin{exercise}
הוכיחו כי הגבול הבא קיים ומצאו את הגבול - 
\[\underset{n\rightarrow\infty}{\lim} \frac{(-1)^n }{n^2}\]
\end{exercise}

\begin{proof}
ברור מהסתכלות על הסדרה שאם יש גבול הוא צריך להיות 0. נראה זאת. לכן יהי 
$\epsilon>0$
ונוכיח שקיים 
$N\in\mathbb{N}$
כך שלכל 
$n\geq N$
מתקיים 
\begin{align*}
|a_n-0|&\leq\epsilon \\
\left|\frac{(-1)^n}{n^2}\right|&\leq \epsilon \\
\frac{1}{n^2}&\leq\epsilon \\
n^2&\geq \frac{1}{\epsilon} \\
n&\geq \sqrt{\frac{1}{\epsilon}} = \epsilon^{-\frac{1}{2}}
\end{align*}
לכן ניקח 
\[N = \left\lceil \epsilon^{-\frac{1}{2}}\right\rceil\]
\end{proof}

\begin{exercise}
הוכיחו לפי ההגדרה כי הסדרה 
\[a_n:=\frac{1+(-1)^n}{2}\]
אינה מתכנסת ל0 ולכן גם אינה מתכנסת כלל.
\end{exercise}

\begin{proof}
על מנת להראות שאין התכנסות ל0 צריך להראות כי קיים 
$\epsilon>0$
כך שלכל 
$N\in\mathbb{N}$
יש 
$n\geq N$
כך ש 
\[|a_n-0|>\epsilon\]
לדוגמה ניקח 
$\epsilon = \frac{1}{2}$
נשים לב כי לכל 
$N\in\mathbb{N}$
יש אינדקס זוגי שגדול יותר ממנו 
$n'$
ומתקיים כי
\[|a_{n'}| = 1>\frac{1}{2}\]
ולכן מתקיים מה שאנחנו רוצים, כלומר הסדרה לא מתכנסת ל0. נניח בשלילה כי יש 
$0\neq L\in\mathbb{R}$
שהוא גבול של הסדרה. באותו אופן נראה עכשיו שזה סתירה בדומה למה שראינו קודם. ניקח 
$\epsilon=\frac{|L|}{2}$ 
זה חיובי כי L חיובי. לכל 
$N\in\mathbb{N}$
קיים אינדקס אי זוגי אשר גדול יותר ממנו 
$n'$
ומתקיים כי
\[|a_n - L| = |0-L| = |L|> \frac{|L|}{2}\]
ולכן נקבל שאין התכנסות לL. 
\end{proof}

\begin{exercise}
הוכיחו לפי ההגדרה כי הסדרה -
\[S_n:=1+\frac{1}{2}+\frac{1}{4}+\dots+\frac{1}{2^n}\]
מתכנסת ומצאו את הגבול.
\end{exercise}

\begin{proof}
נזכור כי הסדרה היא הנדסית ולכן הסכום שמבטא את הסדרה שווה ל -
\[\frac{1\cdot(\left(\frac{1}{2}\right)^n - 1)}{\frac{1}{2}-1} = -2\left(\left(\frac{1}{2}\right)^n -1\right) = 2-\left(\frac{1}{2}\right)^{n-1}\]
לכן הניחוש המושכל יהיה שצריך לבדוק התכנסות ל2. כלומר נרצה להראות כי - 
\[\forall\epsilon>0:\quad \exists N\in\mathbb{N}\quad\forall n\geq N: \quad |S_n-2|<\epsilon\]
 לכן יהי 
$\epsilon>0$
אנחנו רוצים למצוא N מתאים. האי השיוויון הנצרך הוא
\begin{align*}
|S_n-2|&<\epsilon \\ 
\left|\left(\frac{1}{2}\right)^{n-1}\right|<\epsilon \\
\frac{1}{2^{n-1}}<\epsilon  \\
2^{n-1}&>\epsilon^{-1}\\
n-1&>\log \epsilon^{-1}\\
n&>1+\log\epsilon^{-1} 
\end{align*}
ולכן נוכל לקחת 
\[N = \max\left\{\left\lceil 2+\log\epsilon^{-1}\right\rceil, 1\right\}\]
\end{proof}









\end{document}