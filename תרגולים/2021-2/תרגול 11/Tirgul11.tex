\documentclass{article}
\usepackage{fontspec, fullpage}
\usepackage{polyglossia}
\usepackage{amsmath, amssymb, bbm, amsthm, graphicx}
\setmainlanguage{hebrew}
\setmainfont{Times New Roman}
% \newfontfamily{\hebrewfont}{New Peninim MT}
\begin{document}
\title{תרגול 11 חשבון אינפיניטסימלי 1 שנת 2021/2}
\author{ישראל הבר}
\maketitle

\newtheorem{theorem}{משפט}
\newtheorem{lemma}{למה}
\newtheorem{exercise}{תרגיל}
\newtheorem{homeexercise}{תרגיל לבית}
\newtheorem{example}{דוגמה}
\theoremstyle{definition}
\newtheorem{definition}{הגדרה}
\newtheorem{notation}{סימון}
\newtheorem{claim}{טענה}
\newtheorem{comment}{\emph{הערה}}
\renewcommand\qedsymbol{$\blacksquare$}
\newcommand{\limtoinfty}{\underset{n\rightarrow\infty}{\lim}}
\newcommand{\limtur}{\overset{\infty}{\underset{n=1}{\sum}}}
\newcommand{\limturstart}[1]{\overset{\infty}{\underset{n=#1}{\sum}}}
\newcommand{\limsuptoinfty}{\underset{n\rightarrow\infty}{\limsup}}
\newcommand{\liminftoinfty}{\underset{n\rightarrow\infty}{\liminf}}
\newcommand{\limtoinftym}{\underset{m\rightarrow\infty}{\lim}}
\newcommand{\limtop}{\underset{-}{\lim}}
\newcommand{\limbottom}{\overset{-}{\lim}}
\newcommand{\goesto}{\underset{n\rightarrow\infty}{\longrightarrow}}
\newcommand{\goestom}{\underset{m\rightarrow\infty}{\longrightarrow}}
\newcommand{\goesfrom}{\underset{n\rightarrow\infty}{\longleftarrow}}
\newcommand{\funclim}[2]{\underset{#1\rightarrow#2}{\lim}\,}
% \newcommand{\series}{_{n\in\mathbb{N}}}
\newcommand{\series}[2]{\{#1\}_{#2\in\mathbb{N}}}

\section{פתרון הבוחן}
\begin{exercise}
תהי 
$\series{a_n}{n}$.
הוכיחו שאם לכל 
$\epsilon>0$
קיימים 
$M,N$
כך שלכל 
$m>M, n>N$
מתקיים 
$|a_n-a_{m+n}|<\epsilon$,
הסדרה 
$a_n$
מתכנסת.
\end{exercise}

\begin{proof}
יהי 
$\epsilon>0$
על פי הנתון 
\[\exists M>0, \,N>0\forall n>N\quad \forall m>M: |a_n-a_{n+m}|<\epsilon\]
זה כמובן נותן גם כי 
\[\exists M>0, \,N>0\forall n>N\quad \forall m>M\quad\forall k>0: |a_n-a_{n+m+k}|<\epsilon\]
יהיו 
$k_1, k_2>0$
אזי 
\[|a_n-a_{n+m+k_1}|,|a_n-a_{n+m+k_2}|<\epsilon\]
נשתמש באי שיוויון המשולש כדי לקבל 
\[|a_{n+m+k_1} - a_{n+m+k_2}| = |a_{n+m+k_1} - a_n| + | a_{n+m+k_2}-a_n|<\epsilon+\epsilon = 2\epsilon\]
סך הכל נשים לב שזה אומר כי 
\[\forall c_1,c_2>N+M+2:\quad |a_{c_1} - a_{c_2}|<2\epsilon\]
יהי 
$\epsilon_1>0$
ניקח 
$\epsilon = \frac{\epsilon_1}{2}$
לפי מה שמצאנו 
\[\exists N' \in\mathbb{N} \quad\forall n,m>N:\quad |a_n-a_m|<\epsilon_1\]
נשים לב שזה תנאי קושי להתכנסות סדרות ולכן הסדרה מתכנסת
\end{proof}

\bigskip

\begin{exercise}
\begin{enumerate}
\item 
נתונים שני טורים חיוביים 
$\limtur a_n, \limtur b_n$
ו 
$n_0\in\mathbb{N}$
כך שלכל 
$n\geq n_0$
\[\frac{a_{n+1}}{a_n}\leq \frac{b_{n+1}}{b_n}\]
הראו כי אם 
$\limtur b_n$
מתכנס אזי גם 
$\limtur a_n$
מתכנס.
\item 
מצאו התבדרות או התכנסות של הטור הבא 
\[\limtur \frac{n^{n-2}}{e^n n!}\]
\end{enumerate}
\end{exercise}

\begin{proof}
\begin{enumerate}
\item 
נוכל להניח כי 
$n_0=1$
בגלל שהתכנסות טורים תלויה אך ורק בהתכנסות הזנב. לכן מתקיים
\[\forall n\in\mathbb{N}:\quad \frac{a_{n+1}}{a_n}\leq \frac{b_{n+1}}{b_n}\]
נשים לב שזה אומר שבגלל שהטורים חיוביים
\[\forall n\in\mathbb{N}:\quad 0<\frac{a_{n+1}}{b_{n+1}}\leq \frac{a_n}{b_n}\]
לכן נקבל שהסדרה 
$\series{\frac{a_n}{b_n}}{n}$
היא סדרה מונוטונית יורדת, וכמובן חסומה מלרע על ידי 0. לכן הסדרה מתכנסת ל 
$0\leq L\leq\frac{a_1}{b_1}$.
לכן לבסוף 
\[0<\frac{a_n}{b_n}<L+ 1\]
ולכן לבסוף 
\[0<a_n<(L+1)b_n\]
מכיוון שהטור 
$\limtur b_n$
מתכנס גם 
$\limtur (L+1) b_n$
מתכנס. ממבחן ההשוואה הראשון נקבל כי הטור
$\limtur a_n$ 

מתכנס.
\item 
נסמן 
\[a_n = \frac{n^{n-2}}{e^n n!}\]

נשים לב כי 
\begin{align*}
0<\frac{a_{n+1}}{a_n} &=\frac{\frac{(n+1)^{n-1}}{e^{n+1}(n+1)!}}{\frac{n^{n-2}}{e^n\cdot n!}} = \frac{e^n}{e^{n+1}} \cdot \frac{(n+1)^{n-1}}{n^{n-2}}\cdot \frac{n!}{(n+1)!}=e^{-1}\frac{1}{n+1}\frac{(n+1)^{n-1}}{n^{n-2}} \\ 
&= e^{-1} \left(1+\frac{1}{n}\right)^n \left(\frac{n+1}{n}\right)^{-2} 
\end{align*}
ראינו כבר הרצאה כי 
$\left(1+\frac{1}{n}\right)^n\leq e$
ולכן נקבל כי 
\[\frac{a_{n+1}}{a_n}\leq \frac{n^2}{(n+1)^2} = \frac{\frac{1}{(n+1)^2}}{\frac{1}{n^2}} \]
לכן נשים לב שאם נסמן 
$b_n:=n^{-2}$
נקבל כי 
\[\frac{a_{n+1}}{a_n}\leq \frac{b_{n+1}}{b_n}\]
ולכן מכיוון שטורים אלה חיוביים נקבל מסעיף א שאם הטור 
$\limtur b_n$
מתכנס אזי גם 
$\limtur a_n$ 
יתכנס כנדרש. אך נשים לב שראינו בהרצאה כי הטור 
$\limtur b_n$
מתכנס. לכן גם הטור המקורי שלנו יתכנס. 

\end{enumerate}
\end{proof}

\newpage

\begin{exercise}
\begin{enumerate}
\item 
תהי 
$\series{a_n}{n}$
ותהיינה 
$\series{a_{m_n}}{n}, \series{a_{k_n}}{n}$
תתי סדרות שמכסות את כל הסדרה ו 
\[a_{m_n}\goesto a, \quad\quad a_{k_n}\goesto b\]
אפיינו את כל הגבולות החלקיים של 
$a_n$.
\item 
מצאו את כל הגבולות החלקיים של הסדרה הבאה 
\[\begin{cases}(-1)^{n/2}\frac{n}{n+1}, & n=2k \\ \sqrt[n]{\left(\left(1+\frac{1}{n}\right)^n\right)^n + 2^n}+\frac{\sin}{n}, & n=2k-1\end{cases}\]
מותר להשתמש בסעיף א לכל כמות סופית של תתי סדרות מתאימות בלי צורך להוכיח.
\end{enumerate}
\end{exercise}

\begin{proof}
\begin{enumerate}
\item 
יהי גבול חלקי 
$c$
של הסדרה. ניקח תת סדרה המתכנסת לגבול החלקי הזה. נשים לב כי מלפחות אחת התתי סדרות 
\[\series{a_{m_n}}{n}, \series{a_{k_n}}{n}\]
מופיעות אינסוף איברים בתת סדרה זו. ניקח את התת סדרה המתאימה לאיבר זה. תת סדרה זו אם כן תתכנס לגבול של
\[\series{a_{m_n}}{n}, \series{a_{k_n}}{n}\]
בהתאמה. אך מכיוון שסדרה זו היא תת סדרה של סדרה המתכנסת ל
$c$
נקבל ש 
$c$
היא אחת מהגבולות של 
\[\series{a_{m_n}}{n}, \series{a_{k_n}}{n}\]
כלומר 
$c\in\{a, b\}$. 
לכן קבוצת הגבולות החלקיים היא 
$\{a, b\}$
\item 
נשים לב כי ניתן לחלק את הסדרה באופן הבא
\[a_n = \begin{cases}\frac{n}{n+1}, & n=4k \\ -\frac{n}{n+1}, & n=4k-2 \\ \sqrt[n]{\left(\left(1+\frac{1}{n}\right)^n\right)^n + 2^n}+\frac{\sin}{n}, & n=2k-1\end{cases}\]
נשים לב כי עבור ה2 תתי מקרים הראשונים הגבולות החלקיים הם 
$\pm 1$. 
נמצא את הגבול של המקרה השלישי. קודם כל נשים לב כי 
\[-\frac{1}{n}<\frac{\sin}{n}<\frac{1}{n}\]
ומכיוון ש2 הסדרות מימין ומשמאל מכתנסות ל0 נקבל מסנדוויץ' כי 
\[\frac{\sin}{n}\goesto 0\]
עכשיו נמצא את הגבול של 
\[\sqrt[n]{\left(\left(1+\frac{1}{n}\right)^n\right)^n + 2^n}\]
נשים לב כי לפי מה שראינו בהרצאה כי
\[\left(1+\frac{1}{n}\right)^n \geq 2\]
ולכן 
\[\sqrt[n]{\left(\left(1+\frac{1}{n}\right)^n\right)^n}<\sqrt[n]{\left(\left(1+\frac{1}{n}\right)^n\right)^n + 2^n} \leq \sqrt[n]{2\left(\left(1+\frac{1}{n}\right)^n\right)^n}\]
נשים לב כי 2 הסדרות בקצוות האי-שיוויון מתכנסות ל 
$e$
ולכן מסנדוויץ' נקבל כי הסדרה המקורית מתכנסת ל 
$e$.
סך הכל קבוצת הגבולות החלקיים (עם שימוש של סעיף א' למקרה של 3 תתי סדרות) הן 
\[\{\pm 1, e\}\]
\end{enumerate}
\end{proof}






















\end{document}